\documentclass{article}

% if you need to pass options to natbib, use, e.g.:
%     \PassOptionsToPackage{numbers, compress}{natbib}
% before loading neurips_2024


% ready for submission
\usepackage{neurips_2024}

% to compile a preprint version, e.g., for submission to arXiv, add add the
% [preprint] option:
%     \usepackage[preprint]{neurips_2024}


% to compile a camera-ready version, add the [final] option, e.g.:
%     \usepackage[final]{neurips_2024}


% to avoid loading the natbib package, add option nonatbib:
%    \usepackage[nonatbib]{neurips_2024}


\usepackage[utf8]{inputenc} % allow utf-8 input
\usepackage[T1]{fontenc}    % use 8-bit T1 fonts
\usepackage{hyperref}       % hyperlinks
\usepackage{url}            % simple URL typesetting
\usepackage{booktabs}       % professional-quality tables
\usepackage{amsfonts}       % blackboard math symbols
\usepackage{nicefrac}       % compact symbols for 1/2, etc.
\usepackage{microtype}      % microtypography
\usepackage{xcolor}         % colors
\usepackage{graphicx}


\title{Penalty Kick Prediction Model}


% The \author macro works with any number of authors. There are two commands
% used to separate the names and addresses of multiple authors: \And and \AND.
%
% Using \And between authors leaves it to LaTeX to determine where to break the
% lines. Using \AND forces a line break at that point. So, if LaTeX puts 3 of 4
% authors names on the first line, and the last on the second line, try using
% \AND instead of \And before the third author name.


% \author{%
%     David S.~Hippocampus\\
%     Department of Computer Science\\
%     Cranberry-Lemon University\\
%     Pittsburgh, PA 15213 \\
%     \texttt{hippo@cs.cranberry-lemon.edu} \\
% }


\author{%
    Anthony Roca\\
    Department of Computer Science\\
    Wake Forest University\\
    \texttt{rocaaj21@wfu.edu} \\
    \AND 
    Brayden Miller\\
    Department of Computer Science\\
    Wake Forest University\\
    \texttt{millbd21@wfu.edu} \\
    \AND
    Osvaldo Hernandez-Segura\\
    Department of Computer Science\\
    Wake Forest University\\
    \texttt{herno23@wfu.edu} \\  
}

\begin{document}

\maketitle

\begin{abstract} 
  
  This paper presents a novel approach to predicting the direction of a penalty kick in soccer based on the player's mid-swing pose. By leveraging advanced computer vision and deep learning techniques, we aim to provide valuable insights into the decision-making process of penalty takers and potentially assist goalkeepers in positioning themselves more effectively. Our approach involves collecting a large dataset of penalty kick videos, pre-processing the data to extract relevant features, and training a deep neural network model to learn the relationship between the player's pose and the kick direction. We explore transfer learning and data augmentation techniques to improve model performance and address challenges such as player variability and environmental conditions. Our findings demonstrate the potential of our approach for accurate penalty kick prediction, with potential applications in sports analytics, coaching, and betting.
\end{abstract}


\section{Introduction}

\subsection{Background}

The prediction of penalty kick direction has been a topic of interest for researchers and sports enthusiasts for many years. While the outcome of a penalty kick is influenced by various factors, such as player skill, pressure, and luck, understanding the underlying patterns and decision-making processes involved can provide valuable insights.

Early studies on penalty kick prediction focused on traditional statistical analysis and observational methods. Researchers analyzed historical data to identify trends and patterns in player behavior, such as preferred kicking direction or pre-kick routines. However, these approaches were limited by the complexity of human behavior and the difficulty of capturing subtle cues that may influence the player's decision.

With the advancements in computer vision and machine learning, researchers have explored more sophisticated techniques to predict penalty kick direction. These techniques have included:

\begin{itemize}
    \item {\bf Optical Flow Analysis}: Tracking the motion of pixels in the video sequence to capture the dynamics of the player's movement and the ball.
    \item {\bf Feature-Based Methods}: Extracting specific features from images, such as the angle of the player's leg or the direction of their gaze, to provide information about the kick's direction (Figure 1).
    \item {\bf Statistical Models}: Using statistical models to analyze the relationship between player characteristics, environmental factors, and penalty kick outcomes.
\end{itemize}

While these methods have shown some promise, they often struggle to capture the complex patterns and nuances inherent in human movement. Deep learning approaches, with their ability to learn complex representations from large-scale datasets, have emerged as a promising alternative to this problem.

\begin{figure}
    \centering
    \includegraphics[width=0.75\linewidth]{Images/predictionKick.PNG}
    \caption{Kick prediction intuition.}
    \label{fig:enter-label}
\end{figure}


\subsection{Importance of penalty kick prediction}
\begin{itemize}
    \item {\bf Goalkeeper Training}: Accurate predictions can assist goalkeepers in positioning themselves more effectively, increasing their chances of saving penalties. Coaches can use this information to train goalkeepers on specific strategies based on the player's kicking style and tendencies.
    \item {\bf Sports Analytics}: This problem has potential applications in sports analytics, providing valuable insights into the psychology and technique of penalty takers. Analyzing penalty kick data can help identify patterns and trends that could be exploited by opposing teams or used to improve player performance.
    \item {\bf Betting and Gambling}: Accurate predictions could be used by sports bettors to make informed decisions and potentially increase their chances of winning. However, it is important to note that predicting the outcome of any sporting event involves uncertainty and there is no guaranteed strategy for success.
    \item {\bf Television Broadcasting}: Predicting penalty kick direction could be used to enhance the viewing experience for fans by providing real-time analysis and commentary. This could also be used to create interactive features for viewers, such as prediction games or virtual penalty shootouts.
    \item {\bf Academic Research}: Studying the factors that influence penalty kick direction can contribute to a better understanding of human decision-making and motor control. This research can have broader implications for fields such as psychology, sports science, and artificial intelligence.
    \item {\bf Moneyball}: The ability to predict penalty kick direction can be a valuable asset in the context of "Moneyball," a strategy that emphasizes using data analysis and statistical techniques to make informed decisions in sports. By identifying patterns and trends in penalty kick data, teams can develop strategies to improve their performance and gain a competitive advantage.
\end{itemize}

\subsection{Challenges}

\begin{itemize}
    \item {\bf Data Acquisition}: Obtaining a large and diverse dataset of penalty kick sequences with accurate labels can be difficult due to the rarity of penalty shootouts and the need for precise annotations.
    \item {\bf Variability}: Different players have unique kicking styles and techniques, making it challenging to generalize models across various players. This variability introduces complexity and reduces the model's ability to capture common patterns.
    \item {\bf Noise and Occlusions}: Factors like camera angles, player movements, and environmental conditions can introduce noise and occlusions into the data, affecting model performance. These challenges can make it difficult to accurately extract relevant features for prediction.
    \item {\bf Real-time Processing}: Accurate pose estimation in real-time is computationally demanding, especially for complex human poses like a soccer kick. This requires efficient algorithms and powerful hardware to ensure that the model can make predictions promptly.
    \item {\bf Model Generalization}: Models trained on limited datasets may overfit, failing to generalize to unseen data. This means that the model may perform well on the training data but struggle to make accurate predictions on new, unseen data.
    \item {\bf Domain Adaptation}: Models trained on one dataset may struggle to perform well on a different dataset, especially if the distributions are significantly different. This can occur when the data comes from different sources, such as different leagues or time periods.
\end{itemize}



\section{Existing efforts toward our problem}

Research on predicting the direction of a penalty kick based on a player's mid-swing pose has gained significant attention in recent years. Traditional computer vision techniques, such as optical flow analysis and feature-based approaches, have been employed to analyze player movement and extract relevant features. However, these methods often struggle to capture the complex patterns and nuances inherent in human movement.


Deep learning approaches have emerged as a promising alternative for this task. Convolutional Neural Networks (CNNs) have been used to extract features from images of the player mid-swing, while Recurrent Neural Networks (RNNs) have been employed to model the sequential nature of the kick. Graph Neural Networks (GNNs) have also been explored to capture the relationships between different body parts and their influence on the kick direction.


While existing research has made progress in predicting penalty kick direction using machine learning techniques, there is still room for improvement. Deep learning approaches offer promising potential, but further research is needed to address the challenges posed by player variability, environmental conditions, and the complexity of human movement. Additionally, exploring hybrid models that combine CNNs, RNNs, and GNNs may provide even better performance, which is where we look to explore further.




\section{Project plan projection}

{\bf{Phase 1: Data Collection and Pre-Processing (3 Weeks)}}
\begin{itemize}
    \item Collect a large dataset of penalty kick videos with accurate labels from, for example, professional leagues and amateur matches.
    \item Preprocess the data by extracting frames, estimating poses using techniques like OpenPose or AlphaPose, and performing data augmentation (e.g., flipping, rotation).
    \item Handle noise and occlusions using techniques such as data imputation and robust feature extraction.
\end{itemize}

{\bf{Phase 2: Model Development and Training (3-4 Weeks)}}
\begin{itemize}
    \item Experiment with different deep learning architectures, such as CNNs, RNNs, GNNs, or hybrid models that combine them.
    \item Explore transfer learning approaches by fine-tuning pre-trained models on the collected dataset.
    \item Optimize hyperparameters and training procedures using techniques such as grid search or random search.
    \item Evaluate model performance using metrics, for example, accuracy, precision, recall, and F1-score.
\end{itemize}

{\bf{Phase 3: Model Evaluation and Refinement (2-3 Weeks)}}
\begin{itemize}
    \item Conduct thorough evaluation on a separate validation set to assess model generalization.
    \item Identify areas for improvement and iterate on the model architecture, hyperparameters, or data preprocessing techniques.
    \item Address challenges such as overfitting and domain adaptation using regularization techniques or data augmentation.
\end{itemize}

{\bf{Phase 4: Deployment and Real-time Application (3< Weeks)}}
\begin{itemize}
    \item Deploy the final model for real-time applications, integrating it with video processing systems.
    \item Continuously monitor and update the model as new data becomes available to ensure its performance.
    \item Explore potential commercial applications, such as providing real-time predictions during matches or assisting in coaching and training.
\end{itemize}

\section*{References}

\medskip

{
\small

[1] Bransen, L., \& Davis, J. (n.d.). {\it Predicting football penalty directions using in-match performance indicators}. SciSports. https://analytics.scisports.com/research/penalty\_predictor.

[2] Buscà,\ B.,\ Hileno,\ R.,\ Nadal,\ B.,\ \&\ Serna,\ J.\ (2022).\ Prediction of the penalty kick direction in men’s soccer.\ {\it International Journal of Performance Analysis in Sport}, 22(4), 571–582.\ https://doi.org/10.1080/24748668.2022.2097834.

[3] Hunter, A. H., Murphy, S. C., Angilletta, M. J., \& Wilson, R. S. (2018, July 29). {\it Anticipating the Direction of Soccer Penalty Shots Depends on the Speed and Technique of the Kick}. MDPI. https://www.mdpi.com/2075-4663/6/3/73.

[4] Maria, G. D. (2023, January 31).\ {\it Using AI to Predict Penalty Kicks in a Soccer Game}.\ EEWeb. https://www.eeweb.com/how-to-predict-penalty-kicks-in-a-soccer-game-with-artificial-intelligence/.

[5] Salazar, J. A. M., \& Alatrista-Salas, H. (2024, September 11).  {\it Football penalty kick prediction model based on Kicker’s pose estimation: Proceedings of the 2024 9th International Conference on Machine Learning Technologies}. ACM Digital Library. https://dl.acm.org/doi/abs/10.1145/3674029.3674061.

}

\end{document}
